%-----------------------------------------------------------------------------
%
%               Template for sigplanconf LaTeX Class
%
% Name:         sigplanconf-template.tex
%
% Purpose:      A template for sigplanconf.cls, which is a LaTeX 2e class
%               file for SIGPLAN conference proceedings.
%
% Guide:        Refer to "Author's Guide to the ACM SIGPLAN Class,"
%               sigplanconf-guide.pdf
%
% Author:       Paul C. Anagnostopoulos
%               Windfall Software
%               978 371-2316
%               paul@windfall.com
%
% Created:      15 February 2005
%
%-----------------------------------------------------------------------------

\documentclass[nocopyrightspace,blockstyle,numbers]{sigplanconf}

% The following \documentclass options may be useful:

% preprint      Remove this option only once the paper is in final form.
% 10pt          To set in 10-point type instead of 9-point.
% 11pt          To set in 11-point type instead of 9-point.
% numbers       To obtain numeric citation style instead of author/year.

\usepackage{amsmath}
\usepackage[utf8]{inputenc}

\newcommand{\cL}{{\cal L}}

\begin{document}

\special{papersize=8.5in,11in}
\setlength{\pdfpageheight}{\paperheight}
\setlength{\pdfpagewidth}{\paperwidth}

\conferenceinfo{CONF 'yy}{Month d--d, 20yy, City, ST, Country}
\copyrightyear{20yy}
\copyrightdata{978-1-nnnn-nnnn-n/yy/mm}
\copyrightdoi{nnnnnnn.nnnnnnn}

% Uncomment the publication rights you want to use.
%\publicationrights{transferred}
%\publicationrights{licensed}     % this is the default
%\publicationrights{author-pays}

%\titlebanner{banner above paper title}        % These are ignored unless
%\preprintfooter{short description of paper}   % 'preprint' option specified.

\title{Coq's Prolog and application to defining semi-automatic tactics}
%\subtitle{Subtitle Text, if any}

\authorinfo{Théo Zimmermann \and Hugo Herbelin}
           {IRIF, Université Paris Diderot, Université Sorbonne Paris-Cité\\ $\Pi R ^2$, INRIA}
           {theo.zimmermann@univ-paris-diderot.fr / hugo.herbelin@inria.fr}
%\authorinfo{Name1}
%           {Affiliation1}
%           {Email1}
%\authorinfo{Name2\and Name3}
%           {Affiliation2/3}
%           {Email2/3}

\maketitle

\begin{abstract}
  We report on a work-in-progress to re-implement Coq's \texttt{apply}
  tactic in order to embed some form of simple automation. We design
  it in a declarative way, relying on \texttt{typeclasses eauto}, a tactic
  which gives access to the proof-search mechanism behind type classes.
  We qualify this mechanism of ``Coq's Prolog'' and describe it in a
  generic way and explain how it can be used to support the
  construction of automatic and semi-automatic tactics.
\end{abstract}

%\category{CR-number}{subcategory}{third-level}

% general terms are not compulsory anymore,
% you may leave them out
%\terms
%term1, term2

\keywords
apply, automation, Coq, proof assistant, reflection,
type theory, views.

\section{Introduction}

Since version 8.2 \cite{coq86manual}, Coq has included a type class
mechanism which can intervene during type-inference. The resolution
phase of this mechanism is also accessible via the user-side tactic
\texttt{typeclasses eauto}. In fact \texttt{typeclasses eauto}
implements a general Prolog-like proof-search mechanism that can be
instrumented to do much more. While by default, it uses the specific
\texttt{typeclass\_instances} hint database, one can use it with
alternative hint databases. In this case, despite the name of the
tactic, it has nothing to do with the type class mechanism anymore
(apart from the underlying implementation). It can be viewed as a
re-implementation of the former \texttt{eauto} tactic, dating back
from Coq version 5.10.

Many partially automated tactics have already been built upon Coq's type
class mechanism. One example is the re-implementation of the \texttt{rewrite}
tactic \cite{sozeau2010new}, which relies on the mechanism in
a second phase of constraint resolution. Another is our \texttt{transfer}
library \cite{zimmermann}. We argue that both of these applications (and
many others) abuse the type class mechanism, by polluting the
\texttt{typeclass\_\\instances} hint database, when they should instead
define their own disjoint hint databases and rely on the underlying
Prolog-like proof-search mechanism that is made accessible via the
\texttt{typeclasses eauto} tactic.

We demonstrate the usefulness of this ``Coq's Prolog'' with a small
re-implementation of the \texttt{apply} tactic. This tactic is parame\-trized
by a set of ``views'' à la SSReflect \cite{gonthier} which it is
able to apply automatically. It also demonstrates a new way of instrumenting
\texttt{typeclasses eauto} to progress on a goal instead of solving it
(``forward mode'', proposed in particular by Michael Soegtrop
on the Coq-Club mailing list).

\section{Presentation of Coq's Prolog}

The \texttt{typeclasses eauto} tactic is documented in the chapter on
type classes of Coq's user manual \cite[Chapter~20]{coq86manual}.
Like \texttt{eauto}, it is possible to specify on which hint database it
should operate using the \texttt{with myhintdb} clause.
By default, the proof-search depth is unbounded and the traversing
strategy is depth-first. It is also possible to limit the proof-search
depth, and an alternative iterative-deepening traversing strategy exists.

Most hints are lemmas which can be read as Prolog Horn clauses.
For instance:

\begin{verbatim}
Lemma or_introl : forall A B, A -> A \/ B.
\end{verbatim}
can be read as the clause\footnote{We introduce the Prolog predicate
  \texttt{prove} to denote the idea that the hints are used when looking
  for a proof of a statement. When using \text{typeclasses eauto} to
  solve type class goals, there is always a head-constant (the type class)
  which can serve the same role as \texttt{prove} but when using it on
  non-type-class goals, it is not necessarily the case.}:

\begin{verbatim}
prove(or(A,B)) :- prove(A).
\end{verbatim}
Such hints can be introduced with the \texttt{Hint Resolve} command:

\begin{verbatim}
Hint Resolve or_introl : myhintdb.
\end{verbatim}

Some more complex hints can be introduced thanks to the
\texttt{Hint Extern} command. One very special application of
such external hints is to put some sub-goals ``on the shelf'':

\begin{verbatim}
Hint Extern 0 (solveLater _) =>
  unfold solveLater; shelve : myhintdb.
\end{verbatim}

The shelf is a special place in the Coq proof engine which is
normally reserved to sub-goals that will be solved by solving other
sub-goals that depend on them. It can be instrumented to temporarily
store away some sub-goals that we want to keep for later (at which
time they will be ``unshelved'').
The ``unshelve'' tactical can be used to put back in the list of
sub-goals to solve those which had been shelved by the tactic it is
applied on.
Thus, a combination of the previous external hint plus a call to
\texttt{unshelve typeclasses eauto with myhintdb} is a way to use
this Prolog-like proof-search mechanism in forward mode, instead of
the normal solving mode.

Finally, a control operator on the search space is available. It is
not comparable to Prolog's cut operator in that it allows to specify
regular expressions of search-paths which should be cut but does
not restrict backtracking. It is still
limited in that regular expressions can only talk about hints
declared with \texttt{Hint Resolve} and not \texttt{Hint Extern}.
Additionally, there is no negative expressions and even if regular
expressions are known to be closed under complement, the construction
requires to know all the alphabet, which is not the case here since
new hints can be added dynamically. We can expect to see improvements
in control operators for \texttt{typeclasses eauto} in the next versions
of Coq. Here is an example\footnote{Be careful when using it that the
precedence levels are not what one would expect. They should get fixed
in Coq 8.7 but, in the meantime, the best way of writing
forward-compatible code is to parenthesize everything.}
of using this control operator:

\begin{verbatim}
Hint Cut [(_*) or_introl (_*) or_introl] : myhintdb.
\end{verbatim}

\section{An \texttt{apply} tactic with views}

Our work is part of a larger effort to make the life of mathematicians
using Coq easier. The specific issue we address here is to design an
extension of the \texttt{apply} tactic which embeds some bits of
trivial reasoning, such as reasoning modulo symmetry of equations
(identifying \texttt{u = t} and \texttt{t = u}).
There is already a little bit of hard-coded trivial reasoning in the
current implementation (automatic decomposition of single-constructor
inductive types): for instance it supports applying a theorem
of the form \verb|A -> B /\ C| to a goal of the form \texttt{B}.

SSReflect defines views which are basically small theorems of
the form \texttt{A -> B} where \texttt{A} is a different way of
viewing \texttt{B}. Our re-implementation of \texttt{apply} tries to
solve the problem we were describing by allowing the user to
parametrize it with such ``views''. Without any view, it is less
powerful than the current implementation (because of the absence of
decomposition of single-constructor inductive types). With many
views, it can be much more powerful. The various levels of
parametrization could reveal especially useful for teaching (from a
level where everything must be done by hand, to a level where most
details are handled automatically).

Given the inspiration source we described in the previous paragraph,
an obvious application of our work will be to support automatic
insertion of views in the context of small scale reflection. In
particular, it will be possible to simply register the reflection
lemma:

\begin{verbatim}
Lemma andP : forall b1 b2 : bool,
               reflect (b1 /\ b2) (b1 && b2).
\end{verbatim}
in a special hint database and \texttt{apply} will know about it
and use it when necessary.

There are two main ideas in our implementation. The first is that
we are launching a proof search to prove that the theorem
we wish to apply implies the current goal:

\begin{verbatim}
?prove : arrow theorem goal
\end{verbatim}
where \texttt{arrow} is a relation that reifies Coq's implication.
We use it because otherwise the proof-search mechanism would introduce
the premise \texttt{theorem} in the proof context and then try
to prove \texttt{goal} with it, and this is not what we want.

The second is that although we cannot actually prove this implication
most of the time, we can provide an incomplete proof and let the
user fill the holes (this is after all the principle of
\texttt{apply}):

\begin{verbatim}
solveLater A -> arrow B C -> arrow (A -> B) C
\end{verbatim}
where \texttt{solveLater} is a dummy constant introduced to call the
external hint we described earlier, which will shelve the sub-goal
\texttt{A} until the full search succeeds and then unshelve it for the
user to prove. In fact, because the conclusion \texttt{B} might be
dependent on the premise \texttt{A}, we rather write:

\begin{verbatim}
forall (t : T), arrow (U t) V ->
  arrow (forall x : T, U x) V
\end{verbatim}
and then shelve \texttt{t}.
A few other generic rules are present, one of them allowing to
go under quantifiers and, of course, the reflexivity of arrow:

\begin{verbatim}
(forall x : A, arrow (f x) (g x)) ->
  arrow (forall x : A, f x) (forall x : A, g x)

arrow T T
\end{verbatim}

With these rules only, we can reproduce the behavior of
\texttt{apply}, except for the built-in handling of
single-constructor inductive types.
Actually already at that point, we do not reproduce the
exact same behavior because our implementation allows
applying a theorem \texttt{forall x y, P x y}
to a goal \texttt{forall x, P x 0}.

With this basic infrastructure in place, we can start
adding views, such as:

\begin{verbatim}
arrow P P' -> arrow (P /\ Q) P'
arrow Q Q' -> arrow (P /\ Q) Q'
\end{verbatim}
which will be useful in emulating the current behavior
of \texttt{apply}. We can also add a very simple rule to
handle symmetry (one could think of more complex and
powerful ways):

\begin{verbatim}
arrow (u = t) (t = u)
\end{verbatim}
and finally, we register some more generic rules like:

\begin{verbatim}
reflect P b -> arrow P (b = true)
reflect b P -> arrow (b = true) P
\end{verbatim}
so that people can easily extend the hint database with
existing reflection lemmas such as the one seen above.

\section{Conclusion}

This work is very preliminary but shows an interesting path
to removing some of the hindrances there are in using
both SSReflect and vanilla Coq.
It is also a demonstration of the power of
\texttt{typeclasses eauto}, even when not working with
type classes.
Sometimes, the type class mechanism has been used while what
was really wanted was this Coq's Prolog that it gives access to.
One such example is the implementation of \texttt{rewrite}:
it probably could and should be based on a specific hint
database instead of the \texttt{typeclass\_instances} database.

We are planning first to continue to test and improve our
re-implementation of \texttt{apply}. We would in particular like
to have views for applying a theorem modulo commutativity /
associativity. We are also planning to merge this work with our
previous work on applying theorems modulo isomorphisms
\cite{zimmermann}. A lot of ideas from this earlier work were
reused here and the two implementations could very likely be
combined together.

%\appendix
%\section{Appendix Title}

%This is the text of the appendix, if you need one.

%\acks

%Acknowledgments, if needed.

% We recommend abbrvnat bibliography style.

\bibliography{coq-prolog}
\bibliographystyle{abbrvnat}

% The bibliography should be embedded for final submission.

%\begin{thebibliography}{}
%\softraggedright

%\bibitem[Smith et~al.(2009)Smith, Jones]{smith02}
%P. Q. Smith, and X. Y. Jones. ...reference text...

%\end{thebibliography}


\end{document}
